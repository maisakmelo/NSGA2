\documentclass[letterpaper,A4pt]{article}
\usepackage{amsmath}
\usepackage{amssymb}
\usepackage{fancyhdr}
\usepackage{float}
\usepackage[pdftex]{graphicx}
\usepackage[brazil]{babel}
\usepackage[latin1]{inputenc}
\usepackage{amsfonts}
\usepackage{times}
\usepackage[left=1.5cm, right=2cm, top=1cm, bottom=2cm]{geometry}
\usepackage{stackrel}
\begin{document}


\begin{description}
\item[Fun��o 1: Poloni's]

$\stackbin[(x,y)\in\mathbb{R}^2]{}{\mathrm{min}} \left\{%
\begin{array}{rcl}
    f_1(x,y)&=&[1+(A_1-B_1(x,y))^2+(A_2-B_2(x,y))^2] \\
    f_2(x,y)&=&(x+3)^2+(y+1)^2
\end{array}%
\right.$

$-\pi\leq x,y \leq \pi$ 
\vspace{0.5cm}

onde 

\vspace{0.5cm}

$\left\{%
\begin{array}{rcl}
    A_1&=&0.5\sin(1)-2\cos(1)+\sin(2)-1.5\cos(2) \\
    A_2&=&1.5\sin(1)-cos(1)+2\sin(2)-0.5cos(2)\\
    B_1(x,y)&=&0.5\sin(x)-2\cos(x)+2\sin(y)-1.5\cos(y)\\
    B_2(x,y)&=&1.5\sin(x)-\cos(x)+2\sin(y)-0.5\cos(y)
\end{array}%
\right.$

\item[Fun��o 2]

$\stackbin[(x,y)\in\mathbb{R}^2]{}{\mathrm{min}} \left\{%
\begin{array}{rcl}
    f_1(x,y)&=&x \\
    f_2(x,y)&=&\dfrac{1+y}{x}
\end{array}%
\right.\,\,\,\,\,\,\, 0.1\leq x\leq 1, \,\,\, 0\leq y\leq 5$

\item[Fun��o 3: Fonseca e Fleming]

$\stackbin[x\in\mathbb{R}^3]{}{\mathrm{min}} \left\{%
\begin{array}{rcl}
    f_1(x)&=&1-\exp\left[-\displaystyle\sum_{i=1}^3\left(x_i-\dfrac{1}{\sqrt{3}}\right)^2\right]\\
    f_2(x)&=&1-\exp\left[-\displaystyle\sum_{i=1}^3\left(x_i+\dfrac{1}{\sqrt{3}}\right)^2\right]
\end{array}%
\right.$

$ -4\leq x_i\leq 4, \,\,\, i\in\{1,2,3\}$

\item[Fun��o 4: Zitzler-Deb-Thiele]


$\stackbin[x\in\mathbb{R}^n]{}{\mathrm{min}} \left\{%
\begin{array}{rcl}
    f_1(x)&=&x_1\\
    f_2(x)&=&1-\sqrt{\dfrac{f_1(x)}{g(x)}}-\left(\dfrac{f_1(x)}{g(x)}\right)\sin(10\pi f_1(x))
\end{array}%
\right.$

\vspace{0.5cm}

$g(x) = 1+\dfrac{9}{n-1}\displaystyle\sum_{i=2}^nx_i, \,\,\,\,\, 0\leq x_i\leq 1, i\in \{1,2,\cdots,n\}$

\item[Fun��o5: Schaffer N.1] 

$\stackbin[x\in\mathbb{R}]{}{\mathrm{min}} \left\{%
\begin{array}{rcl}
    f_1(x)&=&x^2\\
    f_2(x)&=&(x-2)^2
\end{array}%
\right.,\,\,\,\,\, -A\leq x\leq A$

Valores de $A$ de 10 a $10^5$ t�m sido bem sucedidos.
Valores de $A$ maiores aumentam a dificuldade do problema

\item[Fun��o 6: Schaffer N.2]


$\stackbin[x\in\mathbb{R}]{}{\mathrm{min}} \left\{%
\begin{array}{rcl}
    f_1(x)&=&\left\{%
\begin{array}{rcl}
   -x,&se&x\leq 1\\
    x-2,&se&1<x\leq3\\
    4-x,&se&3<x\leq 4\\
    x-4,&se&x>4
\end{array}%
\right.\\
    f_2(x)&=&(x-2)^2
\end{array}%
\right.,\,\,\,\,\, -5\leq x\leq 10$

\item[Fun��o 7: Kursawe]

$\stackbin[x\in\mathbb{R}^3]{}{\mathrm{min}} \left\{%
\begin{array}{rcl}
    f_1(x)&=&\displaystyle\sum_{i=1}^2\left[-10\exp\left(-0.2\sqrt{x_i^2+x_{i+1}^2}\right)\right]\\
    f_2(x)&=&\displaystyle\sum_{i=1}^3\left[|x_i|^{0.8}+5\sin(x_i^3)\right]
\end{array}%
\right.,\,\,\,\,\, -5\leq x\leq 5, 1\leq i\leq 3$
\end{description}

\end{document}
